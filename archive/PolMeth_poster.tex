% Gemini theme
% See: https://rev.cs.uchicago.edu/k4rtik/gemini-uccs
% A fork of https://github.com/anishathalye/gemini

\documentclass[final]{beamer}

% ====================
% Packages
% ====================
\usepackage{amsfonts}
\usepackage[T1]{fontenc}
\usepackage{lmodern}
\usepackage[size=custom,width=120,height=72,scale=1.0]{beamerposter}
\geometry{paperwidth=48in,paperheight=36in}
\usetheme{gemini}
\usecolortheme{ucf}

\definecolor{harvardred}{rgb}{0.6,0,0}
\definecolor{harvardgray}{RGB}{245,245,245}
\setbeamercolor{block title}{fg=white, bg=harvardred}
\setbeamercolor{block body}{bg=white}
\setbeamercolor{structure}{fg=harvardred}
\setbeamercolor{item}{fg=harvardred}
\setbeamercolor{alerted text}{fg=harvardred}
\setbeamercolor{alerted text in toc}{fg=harvardred}
\setbeamercolor{headline}{bg=harvardred, fg=white}
\setbeamercolor{footline}{bg=harvardred, fg=white}
\definecolor{harvardredlight}{RGB}{248,230,230}
\setbeamercolor{block alerted title}{bg=harvardredlight, fg=black}
\setbeamercolor{block alerted body}{bg=harvardredlight, fg=black}

\definecolor{crimson}{rgb}{0.6,0,0}
\definecolor{darkblue}{HTML}{003366}
\definecolor{graytext}{HTML}{444444}

\usepackage{graphicx}
\usepackage{booktabs}
\usepackage{tikz}
\usepackage{pgfplots}
\pgfplotsset{compat=1.17}
\usepackage{amssymb}
\usepackage{amsfonts}
\usepackage{caption}
\usepackage{subcaption}
\usepackage{setspace}

\documentclass{article}
\usepackage{tikz}
\usetikzlibrary{bayesnet}
\usepackage{xcolor}

% background colors
\definecolor{family1bg}{RGB}{255,245,245}  
\definecolor{family2bg}{RGB}{245,245,255}  
% ====================
% Lengths
% ====================

% If you have N columns, choose \sepwidth and \colwidth such that
% (N+1)*\sepwidth + N*\colwidth = \paperwidth
\newlength{\sepwidth}
\newlength{\colwidth}
\setlength{\sepwidth}{0.025\paperwidth}
\setlength{\colwidth}{0.3\paperwidth}

\newcommand{\separatorcolumn}{\begin{column}{\sepwidth}\end{column}}

% ====================
% Title
% ====================

\title{Dynamic Bipartite Stochastic Blockmodel Regression for Network Data\\ \huge{Application to State and Intergovernmental Organization Networks}}

\author{Qi Liu  \inst{1} \and Ruofan Ma \inst{1} \and Santiago Olivella \inst{3} \and Kosuke Imai \inst{1,2}}

\institute[shortinst]{\inst{1} \textit{Department of Government, Harvard University} \samelineand \inst{2} \textit{Department of Statistics, Harvard University} \samelineand \inst{3} \textit{Department of Political Science, UNC}}

% ====================
% Footer (optional)
% ====================

\footercontent{
  \href{ } {} \hfill
    \hfill
  \href{ }{ }}
% (can be left out to remove footer)

% ====================
% Logo (optional)
% ====================

% use this to include logos on the left and/or right side of the header:
% \logoright{\includegraphics[height=7cm]{groups.png}}
% \logoleft{\includegraphics[height=7cm]{logo2.pdf}}

% ====================
% Body
% ====================

\begin{document}
% \addtobeamertemplate{headline}{}
% {
%     \begin{tikzpicture}[remember picture,overlay]
%       \node [anchor=north west, inner sep=3cm] at ([xshift=0.0cm,yshift=3cm]current page.north west)
%       {\includegraphics[height=8.5cm]{logos/ucf_logo2.png}}; % also try shield-white.eps
%       \node [anchor=north east, inner sep=3cm] at ([xshift=0.0cm,yshift=3cm]current page.north east)
%       {\includegraphics[height=8.5cm]{logos/lab_logo.jpg}};
%     \end{tikzpicture}
% }

\begin{frame}[t]
\begin{columns}[t]
\separatorcolumn

\begin{column}{\colwidth}
  \begin{block}{Motivation}
  
%\begin{itemize}
   % \item Bipartite dynamic networks are common in political science but often misrepresented using static or projected unipartite models, leading to bias and spurious clustering (Olivella et al., 2022; Lo et al., 2023).%\cite{olivella2022dynamic, lo2023statistical}

   % \item We propose a Dynamic Bipartite Mixed-Membership Stochastic Blockmodel (DB-MMSBM) Regression. The model preserves bipartite structure, captures temporal dynamics, and enables interpretable regression-based inference for evolving political networks.

  %  \item Application: State-IGO network. The model shows democratic states cluster around geopolitical alignment, while autocratic states favor material benefits, revealing distinct logics of cooperation.
%\end{itemize}

\begin{itemize}
    \item \textbf{Most political networks are:}
    \begin{itemize}
        \item \textbf{Bipartite} — two node types, ties only across types\\
        \textit{(e.g., states–treaties, legislatures–bills, lobbyists-politicians)}
        \item \textbf{Dynamic} — ties evolve over time
    \end{itemize}

    \item \textbf{Problem:} Use static/ projected unipartite models $\rightsquigarrow$ bias and spurious clustering

    \item \textcolor{crimson}{\textbf{Our contribution: Dynamic Bipartite MMSBM}}

    \begin{figure}
        \centering
        \includegraphics[width=0.9\linewidth]{application/lit.png}

    \end{figure}
    %\begin{itemize}
    %    \item Preserves bipartite structure
    %    \item Captures temporal change
    %    \item Interpretable regression-based inference
    %\end{itemize}

    %\item \textbf{Application:} State–intergovernmental organization network
    %\begin{itemize}
    %    \item Democratic states cluster around geopolitical alignment
    %    \item Autocratic states favor material benefits
    %\end{itemize}

\end{itemize}

  \end{block}

  \begin{block}{Model Setup}
   %\vspace{-0.2in}
\begin{itemize}
  %  \item \textbf{Network Structure}:  $G_t = (V_{1,t}, V_{2,t}, Y_t)$: dynamic bipartite graph at time $t$ %\\($V_{1,t}$, $V_{2,t}$ two disjoint sets of node; $Y_t$: set of edges connecting nodes)
    %Let $G_t = (V_{1,t}, V_{2,t}, E_t)$ be a dynamic bipartite graph observed at time $t$, where $V_{1,t}$ and $V_{2,t}$ are two disjoint sets of nodes, and $E_t$ is the set of edges connecting nodes across the two sets.

  %  \vspace{0.2in}

    \item \textbf{Dynamic bipartite graph $G_t = (V_{1,t}, V_{2,t}, Y_t)$}
    \begin{itemize}
    \item Nodes $p \in V_{1,t}$ and $q \in V_{2,t}$. $Y_{pqt} = 1$ if an edge from $p$ to $q$ exists at time $t$, $Y_{pqt} = 0$ otherwise
        \item $s_t$: latent state; $\mathrm{A}$: transition matrix
  %  \item For each node pair $(p, q)$, where $p \in V_{1,t}$ and $q \in V_{2,t}$, let $z_{pqt} \in \{1, \dots, K_1\}$ be the latent group assignment of node $p$ when interacting with $q$, and $u_{pqt} \in \{1, \dots, K_2\}$ be the latent group of node $q$. Let $y_{pqt} = 1$ if an edge from $p$ to $q$ exists at time $t$, and $y_{pqt} = 0$ otherwise. Mixed-membership vectors are drawn from Markov-dependent Dirichlet distributions with concentration parameters based on node covariates.
\item $\pi_{pt}$, $\psi_{qt}$: mixed membership; $z_{pq,t}$, $u_{pq,t}$: $(p,q)$ interaction specific group indicators
   % \item The model is completed with a $K_1 \times K_2$ block matrix $\mathbf{B}$, where each entry $B_{gh}$ represents the log-odds of an edge forming between latent groups $g$ and $h$. This defines the data-generating process:
   \item  $\mathbf{B}$: $K_1 \times K_2$ block matrix. $B_{gh}$: log-odds of an edge forming between latent groups $g$ and $h$
   \item $\mathbf{x}_{pt}$, $\mathbf{x}_{qt}$: monadic covariates (coefficient: $\boldsymbol{\beta}_{1gm}$, $\boldsymbol{\beta}_{2hm}$); $\mathbf{d}_{p q t}$: dyadic covariates (coefficient: $\boldsymbol{\gamma}$)
 %  \item  $Y_{pqt} = 1$ if an edge from $p$ to $q$ exists at time $t$, $Y_{pqt} = 0$ otherwise
    \end{itemize}

   % \vspace{0.2in}

 
  
\end{itemize} %\vspace{-1cm}

 \begin{figure}[h]
  \centering
  \includegraphics[width=0.9\textwidth]{DAG_horizontal.pdf}
\end{figure}

  %    \begin{itemize} \small
   %     \item $S_{t} \mid S_{t-1}=$ $n \sim$ Categorical $\left(\mathbf{A}_{n}\right)$
   %    \item Given $S_t = m$:
   %     \[
   % \pi_{pt} \sim \text{Dirichlet}\left( \left\{ \exp\left( \mathbf{x}_{pt}^\top \boldsymbol{\beta}_{1gm} \right) \right\}_{g=1}^{K_1} \right),\hspace{-0.9cm} \quad
  %  \psi_{qt} \sim \text{Dirichlet}\left( \left\{ \exp\left( \mathbf{x}_{qt}^\top \boldsymbol{\beta}_{2hm} \right) \right\}_{h=1}^{K_2} \right)
  %      \]
%\item   $z_{pq,t} \sim \text{Categorical}(\pi_{pt}), \quad u_{pq,t} \sim \text{Categorical}(\psi_{qt})$
%\item $Y_{p q t} \sim$ Bernoulli $\left(\text{logit}^{-1}\left(B_{z_{pq,t}, u_{pq,t}}+\mathbf{d}_{p q t}^{\top} \boldsymbol{\gamma}\right)\right) $
 %    \end{itemize}


  \end{block}



\begin{block}{Simulation}

\textbf{Dynamic bipartite networks over 50 periods, 100 nodes per family:}
%\vspace{-0.3cm}
\begin{itemize}
  \item \textbf{Latent states (HMM):} Periods 1–25 in state 1; 26–50 in state 2
  \item \textbf{Covariates:} 
  \begin{itemize}
    \item 1 monadic covariate per family: $x_{pt}, x_{qt} \sim 0.5 \cdot \mathcal{N}(-1.25, 0.09) + 0.5 \cdot \mathcal{N}(1.25, 0.09)$
    \item 1 dyadic covariate: $\mathbf{d}_{pqt} = \mathbf{d}_{pq,1} + \epsilon_{dt}$, where $\epsilon_{dt} \sim \mathcal{N}(0,1)$
  \end{itemize}
  \item \textbf{Difficulty levels:} \textit{Easy, Medium, Hard} — vary $\mathbf{B}$ and $\boldsymbol{\beta}$
  \item \textcolor{crimson}{\textbf{Model recovers:}}
  \begin{itemize}
    \item \textcolor{crimson}{\textbf{Mixed-membership vectors, group structure, regression parameters}}
    \item \textcolor{crimson}{\textbf{Good performance across difficulty levels}}
  \end{itemize}
\end{itemize}

\end{block}



\end{column}

\separatorcolumn

\begin{column}{\colwidth}
\begin{table}[ht]
\centering
\label{tab:parameter_matrices}
\[
\begin{array}{c|c|c|c}
    & \textbf{Easy} & \textbf{Medium} & \textbf{Hard} \\
    \hline
    \text{logit}^{-1}(\mathbf{B}) & 
    \begin{bmatrix} 0.90 & 0.01 \\ 0.10 & 0.60 \end{bmatrix} & 
    \begin{bmatrix} 0.90 & 0.05 \\ 0.25 & 0.60 \end{bmatrix} & 
    \begin{bmatrix} 0.90 & 0.10 \\ 0.40 & 0.60 \end{bmatrix} \\[2ex]
    \beta_1 & 
    \begin{bmatrix} 1.50 & -0.50 \\ 1.25 & -1.50 \end{bmatrix} & 
    \begin{bmatrix} 1.00 & -0.25 \\ 0.75 & -1.00 \end{bmatrix} & 
    \begin{bmatrix} 0.60 & -0.05 \\ 0.50 & -0.55 \end{bmatrix} \\[2ex]
    \beta_2 & 
    \begin{bmatrix} -3.00 & 1.00 \\ 7.25 & -4.25 \end{bmatrix} & 
    \begin{bmatrix} -0.50 & 0.25 \\ 1.25 & -1.25 \end{bmatrix} & 
    \begin{bmatrix} -1.05 & -0.55 \\ 1.55 & -0.25 \end{bmatrix} \\
\end{array}
\]
\caption*{\normalsize Easy, medium, to hard DGPs: more similar entries in $\mathbf{B}$ and more mixed memberships}
%{\footnotesize Note: Columns: increasing complexity. Rows: blockmodels ($\mathbf{B}$) and regression coefficient vectors ($\boldsymbol{\beta}$) by HMM state. \\Easy to hard: more similar entries in $\mathbf{B}$ and more mixed memberships.}
\end{table}

{\begin{center}
    \textbf{\underline{Medium Case Results}:}
\end{center}} 

   \begin{figure} \footnotesize
                \begin{center}
                \begin{tabular}{ccc}
                 \includegraphics[width=0.33\linewidth]{simulation/mod_S_all.pdf}&
                 \includegraphics[width=0.33\linewidth]{simulation/mod_B_all.pdf}&
                 \includegraphics[width=0.33\linewidth]{simulation/bm.pdf}\\
              a) Mixed-membership (family 1) & b) Mixed-membership (family 2) & c) Blockmodel  \\
                  \end{tabular}
                  \end{center}
                  \caption*{\normalsize a) and b): estimated mixed-membership vectors align with known values; \\
                  c): estimated blockmodel match known values (white numbers)} \label{mm_B}
                %  Note: a) and b): estimated mixed-membership vectors by time periods against known values. \\ c): estimated blockmodel against known values (indicated by the white numbers in each cell).  \label{mm_B}
            \end{figure}

 
   \begin{figure} \footnotesize
                \begin{center}
                \begin{tabular}{ccc}
                 \includegraphics[width=0.33\linewidth]{simulation/pred_prob_all.pdf}&
                 \includegraphics[width=0.33\linewidth]{simulation/hess_S.pdf}&
                 \includegraphics[width=0.33\linewidth]{simulation/hess_B.pdf}\\
              a) Mixed-membership prediction with $\hat{\beta}$ & b) SE of $\hat{\beta}$ (family 1)  & c) SE of $\hat{\beta}$ (family 2)  \\
                  \end{tabular}
                  \end{center}
                  \caption*{\normalsize a): mixed-membership predicted using estimated coefficients match prediction using true coefficients.\\
                  b) and c): SEs across 100 simulated networks covers sd($\hat{\beta}$) (red crosses) well.}\label{beta}
                 % Note: a): mixed-membership predicted using estimated coefficients vs using true coefficients.\\b) and c): SEs across 100 simulated networks vs sd($\hat{\beta}$) (red crosses).  \label{beta}
            \end{figure}




\begin{block}{Application: State-IO Network, 1965-2014}

\begin{itemize}
    \item \textbf{Data}: Yearly state-IGO membership data  between 1965-2014, covering 200 countries and regions and 471 IGOs (Pevehouse et al., 2020; Davis \& Pratt, 2021). %\cite{pevehouse2020tracking, davis2021forces}

    \item \textbf{Parameters}: Three groups for both the state (\textbf{S}) and IGO (\textbf{I}) families. One latent state.
\end{itemize}
    
\end{block}

\begin{alertblock}{Group Labels Based on Estimated Block Model}
\begin{spacing}{1.5}
      \begin{itemize} \normalsize
        \item \textbf{\textcolor{crimson}{S1 (Internationalists)}}: Most likely to instantiate links to a large number of IGOs in I1 and I3.
        \item \textbf{\textcolor{crimson}{S2 (Opportunists)}}: Most populus group in the state family. Very likely to instantiate links to IGOs in I1, unlikely to interact with I2 and I3.
        \item\textbf{\textcolor{crimson}{S3 (Isolationists)}}: A small number of states that has a low likelihood to interact with IGOs across groups.
        
        \vspace{1in}
        
        \item \textbf{\textcolor{crimson}{I1 (Universal IGOs)}}: Attract a large number of states across groups.
        \item \textbf{\textcolor{crimson}{I2 (Trivial IGOs)}}: Large number of IGOs that are unlikely to instantiate links with any groups of state.
        \item \textbf{\textcolor{crimson}{I3 (Exclusive IGOs)}}: A small number of IGOs that only interact with states in S1.
    \end{itemize}
    \end{spacing}
\end{alertblock}

  


\end{column}

\separatorcolumn

\begin{column}{\colwidth}


\begin{figure} \footnotesize
    \centering
    \includegraphics[width=0.7\linewidth]{application/3by3_hess.pdf}
    %\caption{Network graph summarizing the estimated blockmodel, where size of the nodes (circles) reflects aggregate membership in each group and weighted edges (lines) reflect the probability of membership.}
\end{figure}

\begin{block}{Validating Group Labels}

 The \textcolor{crimson}{\textbf{covariate effects align with our group labels}}. For instance:
 
 \begin{itemize}
     \item Rich, democratic states that are geo-politically aligned with the US are more likely to have larger share of mixed-membership in S1.

         \item IGOs whose average members are more US-aligned and cross-regional are more likely to have a larger share of mixed-membership in I3. 
 \end{itemize}


    \begin{table}
      \normalsize \centering
    \begin{tabular}{lcccccc}
    \midrule \midrule \\[-1.8ex] Predictor & S1 & S2 & S3 &  I1 & I2 & I3 \\ 
    \midrule UN IP  & -0.59 & -2.26 & -0.83 && \\ 
      & (0.29) & (0.26) & (0.25) && \\ 
    V-Dem  & 2.95 & 3.97 & 0.68 &&\\
    & (0.63) & (0.72) & (0.71) &&\\
    GDPpc  & 0.25 & 0.02 & -0.05 && \\ 
    & (0.07) & (0.06) & (0.08) &&\\
    Europe  & 1.33 & -2.61 & 0.37 && \\
    & (0.50) & (0.47) & (0.46) && \\
    \midrule 
    Ideal point (lagged) &&&& -2.70 & -2.73 & -0.39 \\
    & & & & (0.29) & (0.28) & (0.25) \\
    Regional IO & & & & 0.72 & 0.80 & -0.05 \\ 
    & & & & (0.51) & (0.57) & (0.53) \\
    Mem. Size (lagged) & & & & 0.00 & -0.05 & -0.01 \\
     & & & & (0.01) & (0.01) & (0.01) \\
    Salient IO  & & & & -0.16 & -0.12 & -0.23 \\
    & & & & (0.57) & (0.57) & (0.63) \\
    \midrule Number of Dyads & 1,833,000 & & & & &  \\ \midrule \midrule
    \end{tabular}
      %\caption{Estimated effects of monadic coefficients. The estimates represent the effect of each covariate on the log-odds of membership in each latent group. Standard errors are included in the parentheses.}
    \end{table}

\vspace{0.5in}
\textcolor{crimson}{\textbf{Dynamic changes in mixed-membership align with related political events}}:

    
 \begin{figure} \footnotesize
                \begin{center}
                \begin{tabular}{ccc}
                 
                 \includegraphics[width=0.88\linewidth]{application/state_mem_hess.jpg}&
                 
                  \end{tabular}
                  \end{center}
                  %\caption{Dynamic History of Selected States' Mixed-Membership in G1-G3.}
            \end{figure}

            
\end{block}





\end{column}

\separatorcolumn
\end{columns}
\end{frame}

\end{document}
